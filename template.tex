\documentclass{article}

\usepackage{listings}
\usepackage{fullpage}
\usepackage{color}


% Colori per inserire syntax highlight

\definecolor{dkgreen}{rgb}{0,0.6,0}
\definecolor{gray}{rgb}{0.5,0.5,0.5}
\definecolor{mauve}{rgb}{0.58,0,0.82}

% Setting per Java

\lstset{frame=tb,
  language=Java,
  aboveskip=3mm,
  belowskip=3mm,
  showstringspaces=false,
  columns=flexible,
  basicstyle={\small\ttfamily},
  numbers=none,
  numberstyle=\tiny\color{gray},
  keywordstyle=\color{blue},
  commentstyle=\color{dkgreen},
  stringstyle=\color{mauve},
  breaklines=true,
  breakatwhitespace=true,
  tabsize=3
}
%
% Settings per il TITOLO
%
\author{Guglielmo Bartelloni}
\title{Titolo}

%
% Inizio del documento
%
\begin{document}
\maketitle
\tableofcontents
\clearpage
%
% Inizio del CORPO
%
\section{Scopo dell'esercitazione}
%
% Per inserimento del programma Java
%
\begin{lstlisting}
\end{lstlisting}

\section{Cenni Storici}

\subsection{Argomento}


\section{Analisi Funzionale}

\subsection{Ipotesi Risolutiva}

\subsection{Funzionalita' del programma}


\section{Analisi Tecnica}

\subsection{Scomposizione Top-Down}

\subsection{UML}

\subsection{Descrizione Classi}
\subsubsection{Classe1}
\subsubsection{Classe2}


\section{Test Data Set/Debug}

\end{document}
