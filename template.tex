\documentclass[a4paper,12pt,times,numbered,print,index]{article}

\usepackage[italian]{babel}
\usepackage[utf8]{inputenc}
\usepackage{graphicx}
\usepackage{listings}
\usepackage{color}
\usepackage{fancyhdr}
\usepackage[margin=1in]{geometry}
\usepackage{hyperref}
\usepackage[backend=biber, style=authortitle-comp]{biblatex}
%
% File della bibliografia
%
\addbibresource{bibliografia.bib}

%
% Impostazioni per head e foot
%
\pagestyle{fancy}
\fancyhead{}
\fancyfoot{}
\fancyhead[R]{<++>}
\fancyfoot[L]{\thepage}
\fancyfoot[R]{\slshape{\footnotesize{Guglielmo Bartelloni}}}

%
% Impostazioni per link dell'indice
%
\hypersetup{
    colorlinks,
    citecolor=black,
    filecolor=black,
    linkcolor=black,
    urlcolor=black
}

% Colori per inserire syntax highlight
\definecolor{dkgreen}{rgb}{0,0.6,0}
\definecolor{gray}{rgb}{0.5,0.5,0.5}
\definecolor{mauve}{rgb}{0.58,0,0.82}

% Setting per Java

\lstset{frame=tb,
  language=Java,
  aboveskip=3mm,
  belowskip=3mm,
  showstringspaces=false,
  columns=flexible,
  basicstyle={\small\ttfamily},
  numbers=none,
  numberstyle=\tiny\color{gray},
  keywordstyle=\color{blue},
  commentstyle=\color{dkgreen},
  stringstyle=\color{mauve},
  breaklines=true,
  breakatwhitespace=true,
  tabsize=3
}
\author{Guglielmo Bartelloni}
%
% Inizio del documento
%
\begin{document}
%
% TITOLO
%
\begin{titlepage}
\begin{center}
	\vspace{1cm}
	\textbf{\huge{<++>}}\\ 
	\vspace{1cm}
	\includegraphics[scale=0.7]{logoITTS.jpg}\\
	\vspace{1cm}
	\large{Guglielmo Bartelloni}\\
	\vspace{0.5cm}
	24 Dicembre, 2017\\
	\today\\
	\vspace{0.5cm}
	\vspace{0.5cm}
	4IB\\
	Luigi Vestri e Davide Caramelli\\
	\vspace{1cm}
	\Large{Laboratorio di Informatica}
\end{center}
\end{titlepage}
\vspace*{1cm}
\tableofcontents
\clearpage
%
% Inizio del CORPO
%
\section{Scopo dell'esercitazione}
La programmazione orientata agli oggetti e' un tipo di programmazione particolare perche' permette di rappresentare oggetti reali all'interno di un programma

\textcite{oggetti}
ciao

%
% Per inserimento del programma Java
%
\begin{lstlisting}
\end{lstlisting}

\section{Cenni Storici}

\subsection{Argomento}


\section{Analisi Funzionale}

\subsection{Ipotesi Risolutiva}

\subsection{Funzionalita' del programma}


\section{Analisi Tecnica}

\subsection{Scomposizione Top-Down}

\subsection{UML}

\subsection{Descrizione Classi}
\subsubsection{<++>} %Nome della classe 
\subsubsection{<++>} %Nome della classe


\section{Test Data Set/Debug}

\printbibliography
\end{document}
